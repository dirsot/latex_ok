\chapter{State of art}
\label{cha:State of art}
20-30
W rozdziale tym przedstawiono podstawowe informacje dotyczące struktury prostych plików \LaTeX a. Omówiono również metody kompilacji plików z zastosowaniem programów \emph{latex} oraz \emph{pdflatex}.

%---------------------------------------------------------------------------

\section{Rodzaje map}
\label{sec:Rodzaje map}

\subsection{Google Maps}
\label{subsec:Google Maps}

\subsection{Windows Maps}
\label{subsec:Windows Maps}

\subsection{Apple Maps}
\label{subsec:Apple Maps}

\section{Google Earth}
\label{sec:Google Earth}

\section{Time line}
\label{sec:Time line}

Plik \LaTeX owy jest plikiem tekstowym, który oprócz tekstu zawiera polecenia formatujące ten tekst (analogicznie do języka HTML). Plik składa się z dwóch części:
\begin{enumerate}%[1)]
\item Preambuły -- określającej klasę dokumentu oraz zawierającej m.in. polecenia dołączającej dodatkowe pakiety;

\item Części głównej -- zawierającej zasadniczą treść dokumentu.
\end{enumerate}


\begin{lstlisting}
\documentclass[a4paper,12pt]{article}      % preambuła
\usepackage[polish]{babel}
\usepackage[utf8]{inputenc}
\usepackage[T1]{fontenc}
\usepackage{times}

\begin{document}                           % część główna

\section{Sztuczne życie}

% treść
% ąśężźćńłóĘŚĄŻŹĆŃÓŁ

\end{document}
\end{lstlisting}

%---------------------------------------------------------------------------

\section{Kompilacja}
\label{sec:kompilacja}

\begin{lstlisting}
latex test.tex
dvips test.dvi -o test.ps
ps2pdf test.ps
\end{lstlisting}
%
lub za pomocą PDF\LaTeX:
\begin{lstlisting}
pdflatex test.tex
\end{lstlisting}

%---------------------------------------------------------------------------

\section{Narzędzia}
\label{sec:narzedzia}

\begin{itemize}
\item Edit shortcuts -- definiowanie własnych klawiszy skrótu;
\item Line Tools -- dodatkowe operacje na liniach tekstu;
\end{itemize}

%---------------------------------------------------------------------------

\section{Przygotowanie dokumentu}
\label{sec:przygotowanieDokumentu}

Plik źródłowy \LaTeX a jest zwykłym plikiem tekstowym. Przygotowując plik
źródłowy warto wiedzieć o kilku szczegółach:

\begin{itemize}
\item
Poszczególne słowa oddzielamy spacjami, przy czym ilość spacji nie ma znaczenia.
Po kompilacji wielokrotne spacje i tak będą wyglądały jak pojedyncza spacja.
Aby uzyskać {\em twardą spację}, zamiast znaku spacji należy użyć znaku {\em
tyldy}.

\item
Znakiem końca akapitu jest pusta linia (ilość pusty linii nie ma znaczenia), a
nie znaki przejścia do nowej linii.

\item
\LaTeX~sam formatuje tekst. \textbf{Nie starajmy się go poprawiać}, chyba, że
naprawdę wiemy co robimy.
\end{itemize}


